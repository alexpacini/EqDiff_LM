Partendo dalle leggi generali (conservazione, bilancio di massa, energia, ecc) e dalle leggi costitutive si andranno a definire
i vari modelli matematici composti dall'equazione (o sistema di equazioni) alle derivate parziali che governano i vari fenomeni fisici associati.
Attraverso l'imposizione di condizioni iniziali e/o condizioni al contorno si dimostra l'esistenza, l'unicit\`a della soluzione e la dipendenza continua
dai dati iniziali.
%%%%%%%%%%%%%%%%%%%%%%%%%%%%%%%%%%%%%%%%%%%%%%%%%%%%%%%%%%%%%%%%%%%
\section{Classificazione delle Equazioni di II ordine in due variabili ($t,x$)}
La forma completa di un' equazioni di II ordine in due variabili pu\`o essere espressa come segue
\[
	\underbrace{au_{tt}+2bu_{xt}+cu_{xx}}_\text{parte principale} + 
	du_t + eu_x + hu = f
\]
con $a>0$.
Considerando quindi la parte principale e sostituendo la derivata rispetto a $t$ con la variabile simbolica $p$ mentre la derivata 
rispetto a $x$ con $q$, si pu\`o scrivere
\[
	ap^2 + 2bpq + cq^2 = tr(AH)
\]
dove $A$ \`e la matrice associata all'equazione differenziale e $H$ \`e la matrice Hessiana di $u$.
\[
A=
 \begin{pmatrix}
  a & b \\
  b & c
 \end{pmatrix}
\;\;\;
%%%%%%%%%%%%%%%%%%%%%%%%%%%%%%%%%%%%%%%%%%%%%%%%%%%%%%%%%%%%%%%%%%%
H=
 \begin{pmatrix}
  \partial_{tt} & \partial_{tx} \\
  \partial_{xt} & \partial_{xx}
 \end{pmatrix}
\]
\`E ora possibile classificare le equazioni differenziali in base alla matrice $A$.
%
\begin{align*}
& A \mbox{ indefinita }  \Rightarrow  \mbox{ iperbolica}\\
& A \mbox{ semidefinita positiva }  \Rightarrow  \mbox{ parabolica}\\
& A \mbox{ definita positiva }  \Rightarrow  \mbox{ ellittica}
\end{align*}
%
Infatti, definito $\Delta=b^2-4ac$, se
%
\begin{align*}
& \Delta>0 \Rightarrow tr(AH)=1 \mbox{ indica un iperbole }\\
& \Delta=0 \Rightarrow tr(AH)=1 \mbox{ indica una parabola }\\
& \Delta<0 \Rightarrow tr(AH)=1 \mbox{ indica un'ellisse }
\end{align*}
%
La classificazione si estende in maniera naturale ad equazioni in $n>2$ variabili.

{\bf Esempi noti}
\[
\begin{array}{ll}
	u_t - Du_{xx}=f &\mbox{ eq. della diffusione: parabolica }\\
	u_{tt} + u_{xx}=f &\mbox{ eq. di Laplace: ellittica }\\
	u_{tt} - c^2 u_{xx}=f &\mbox{ eq. delle onde: iperbolica }
\end{array}
\]
Un'equazione pu\`o anche essere di tutti e tre i tipi, ne \`e un esempio l'equazione di Eulero-Tricomi ($u_{tt}-tu_{xx}=f$), che per $t>0$ \`e iperbolica,
per $t=0$ parabolica e per $t<0$ ellittica. 