\section{Derivazione dell'equazione del calore}
Definiamo innanzitutto le variabili in gioco:\\
$t=$ tempo, $x=$ posizione, $u(t,x)=$ temperatura nella posizione $x$ e al tempo $t$.\\
Nel definire il modello si far\`a uso di:
%
\begin{align*}
& r= \mbox{ tasso di calore per unit\`a di massa dall'esterno } \; [r]=\frac{[cal]}{[tempo][massa]}\\
& \rho= \mbox{ densit\`a (lineare) di massa della barra } \; [\rho]=\frac{[massa]}{[lunghezza]}\\
& q= \mbox{ flusso di calore } \; [q]=\frac{[cal]}{[tempo]}\\
& e= \mbox{ energia interna per unit\`a di massa } \; [r]=\frac{[cal]}{[massa]}\\
\end{align*}

Il primo passo nella derivazione dell'equazione del calore consiste nell'applicare la \textit{Legge di Bilancio}:\\
isolata una porzione $[x_0, x_0+h]$ della barra, il tasso di variazione dell'energia interna eguaglia il flusso agli estremi;
nel caso di sorgente, il tasso di variazione del calore erogato sar\`a sommato al flusso agli estremi.
\[
	\underbrace{\frac{d}{dt}\int_{x_0}^{x_0+h} e(t,x)\rho dx}_\text{Variazione dell'energia rispetto al tempo}
	= \overbrace{q(t,x_0)-q(t,x_0 +h)}^\text{Flusso entrante}
	+\underbrace{\int_{x_0}^{x_0+h} r(t,x) \rho dx}_\text{Flusso della sorgente}
\]
Per il Teorema Fondamentale del Calcolo Integrale
\[
	q(t,x_0)-q(t,x_0 +h) = -\int_{x_0}^{x_0+h} q_x(t,x)dx
\]
dove $q_x$ indica $\frac{dq}{dx}$.

Considerando che l'espressione 
\[
	\frac{d}{dt}\int_{x_0}^{x_0+h} e(t,x)\rho dx
	= -\int_{x_0}^{x_0+h} q_x(t,x)dx
	+\int_{x_0}^{x_0+h} r(t,x) \rho dx
\]
deve essere valida per $x_0$ e $x_0+h$ e che, data la continuit\`a dell'energia \`e possibile portare 
la derivata all'interno del segno di integrale, si ottiene la Legge di Bilancio in forma locale
\[
	\frac{\partial}{\partial t} e(t,x)\rho= -\frac{\partial}{\partial x}q(t,x)+\rho r(t,x)
\]

\`E ora necessario applicare le leggi costitutive, che risultano essere delle leggi sperimentali.\\
La prima, che prende il nome di \textit{Legge di Fourier}, indica che il flusso di calore ($q$) \`e direttamente proporzionale
alla derivata spaziale della temperatura secondo la legge
\[
	q= -ku_x
\]
con $u=u(t,x)$ e $k>0$. Il segno negativo indica che si ha il flusso positivo passando dalla zona pi\`u calda a quella pi\`u fredda.
\[
	[k]= \frac{[cal]}{[tempo]}\frac{[lunghezza]}{[grado]}
\]
La seconda lega invece l'energia alla temperatura
\[
	e= c_lu
\]
dove $c_l$ indica il calore specifico ed \`e $>0$
\[
	[c_l]=\frac{[cal]}{[massa][grado]}
\]
Operando la sostituzione si ottiene
\[
	\rho c_l \frac{\partial}{\partial t} u= k \frac{\partial^2}{\partial x^2}u + \rho r
\]
che riordinata
\[
	u_t= \underbracket{D}_{\mathclap{\text{Risposta termica}}} u_{xx}+f
\]
dove $D=k/c_l\rho$ e $f=r/c_l$.
\[
	[D]=\frac{\cancel{[cal]}[lunghezza]}{[tempo]\cancel{[grado]}}\frac{\cancel{[massa]}\cancel{[grado]}}{\cancel{[cal]}}
	\frac{[lunghezza]}{\cancel{[massa]}}=\frac{[lunghezza]^2}{[tempo]}
\]

L'equazione caratteristica risulta quindi essere, considerata l'equazione differenziale omogenea e sostituendo due variabili algebriche alle due variabili derivate
\[
	u_t= Du_{xx} \;\;\; \Rightarrow \;\;\; T=DX^2
\]
Si noti che \`e l'equazione di una parabola.

%%%%%%%%%%%%%%%%%%%%%%%%%%%%%%%%%%%%%%%%%%%%%%%%%%%%%%%%%%%%%%%%%%%%%%%%%%%%%%%%%%%%%%%%%%%%%%%%%%%%%%%%%%%
\section{Problemi ``Ben Posti''}
Si considerino i cosiddetti ``problemi ben posti'', essi saranno del tipo
\[
	\left\{
	\begin{array}{ll}
		u_t=Du_{xx} & x\in\mathbb{R}, \; 0<t<T \\
		u(0,x)=g(x) & \text{Temperatura Iniziale}\\
		u(t,0)=\alpha(t) \; , \; u(t,L)=\beta(t) & \text{Condizioni di Dirichlet agli estremi}\\
		u_x(t,0)=\alpha(t) \; , \; u_x(t,L)=\beta(t) & \text{Condizioni di Neumann agli estremi}
	\end{array}
	\right.
\]
Le condizioni di Dirichlet corrispondono a fissare la temperatura sui capi della sbarra, mentre con Neumann si fissa
il flusso (condizioni di Neumann nulle significano che la barra \`e isolata agli estremi). Non sono state poste condizioni per $t=T$
per la causalit\`a del sistema in esame.

\section{Unicit\`a e dipendenza continua dai dati}
Si inizia con il considerare la temperatura sulla barra
\[
	E(t)= \frac{1}{2}\int_0^L u^2 (t,x) dx
\]
e la si deriva rispetto al tempo
\[
	E'(t)= \frac{1}{2}\int_0^L 2u(t,x)u_t(t,x)dx
\]
Nel passaggio precedente ci si \`e posti nella condizione in cui la derivata della somma equivale alla somma delle derivate.\\
Considerando ora $u_t=Du_{xx}$ si ottiene
\[
	E'(t)= D\int_0^L u(t,x)u_{xx}(t,x)dx
\]
Ora, utilizzando l'integrazione per parti
\[
	D\left[u(t,x)u_x(t,x)\right]_0^L - D\int_0^L 
	\underbrace{u_x(t,x)u_x(t,x)}_{u_x^2(t,x)}dx
\]
Nel caso di condizioni agli estremi nulle si ha $u(t,0)=u(t,L)=0$, perci\`o il termine $D\left[u(t,x)u_x(t,x)\right]_0^L=0$ e quindi
\[
	E'(t)= - D \int_0^L u_x^2(t,x) dx \leq 0
\]
La derivata negativa indica che $E(t)\leq E(0)$, segue che
\[
	\int_0^L u^2(t,x)dx \leq \int_0^L g^2 (x) dx
\]
Perci\`o, considerato il sistema privo di ingressi e quindi l'equazione
omogenea, l'energia non aumenta.\\
Si consideri nuovamente l'equazione $u_t=Du_{xx}$; essa \`e lineare, perci\`o
se $u_1$ e $u_2$ sono soluzioni e $C_1,\; C_2$ costanti, anche $u=C_1u_1+C_2u_2$ \`e soluzione.\\
Se
\[
	\underbracket{
		\begin{array}{l}
			u_1 \text{ \`e soluzione con temperatura iniziale } g_1 \\
			u_2 \text{ \`e soluzione con temperatura iniziale } g_2
		\end{array}
		}_{\Downarrow}
\]
\[
	u_1-u_2 \text{ \`e soluzione con temperatura iniziale } g_1-g_2
\]
che applicato alla disuguaglianza precedente
\[
	\int_0^L \left(u_1(t,x)-u_2(t,x)\right)^2 dx
	\leq
	\int_0^L \left(g_1(x)-g_2(x)\right)^2 dx
\]
che garantisce:\\
{\bf Unicit\`a}: Se $g_1=g_2$ si ottiene
\[
	\int_0^L \underbrace{\left(u_1(t,x)-u_2(t,x)\right)^2}_\text{sempre positivo o nullo} dx
	\leq 0
\]
essendo la somma di quadrati sempre positiva
\[
	\int_0^L \left(u_1(t,x)-u_2(t,x)\right)^2 dx
	= 0
\]
e quindi $u_1(t,x)=u_2(t,x)$
Perci\`o con le stesse condizioni iniziali si ottiene la stessa soluzione.\\
{\bf Dipendenza continua della soluzione dai dati}:\\
Una differenza infinitesima nelle condizioni iniziali comporta una differenza
infinitesima nella soluzione, garantendo la continuit\`a dai dati (non diverge).

\section{Problema di Cauchy globale}
Riprendendo il problema di Dirichlet con condizioni nulle agli estremi
\[
	\left\{
	\begin{array}{l}
		u_t=Du_{xx} \\
		u(0,x)=g(x) \\
		u(t,0)=0, \; u(t,L)=0
	\end{array}
	\right.
\]
Svincoliamo ora la soluzione da $g(x)$ (sar\`a ripreso successivamente)
\[
	\left\{
	\begin{array}{l}
		u_t=Du_{xx} \\
		u(t,0)=0, \; u(t,L)=0
	\end{array}
	\right.
\]
