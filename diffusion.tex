\section{Derivazione dell'equazione del calore}
Definiamo innanzitutto le variabili in gioco:\\
$t=$ tempo, $x=$ posizione, $u(t,x)=$ temperatura nella posizione $x$ e al tempo $t$.\\
Nel definire il modello si far\`a uso di:
%
\begin{align*}
& r= \mbox{ tasso di calore per unit\`a di massa dall'esterno } \; [r]=\frac{[cal]}{[tempo][massa]}\\
& \rho= \mbox{ densit\`a (lineare) di massa della barra } \; [\rho]=\frac{[massa]}{[lunghezza]}\\
& q= \mbox{ flusso di calore } \; [q]=\frac{[cal]}{[tempo]}\\
& e= \mbox{ energia interna per unit\`a di massa } \; [r]=\frac{[cal]}{[massa]}\\
\end{align*}

Il primo passo nella derivazione dell'equazione del calore consiste nell'applicare la \textit{Legge di Bilancio}:\\
isolata una porzione $[x_0, x_0+h]$ della barra, il tasso di variazione dell'energia interna eguaglia il flusso agli estremi;
nel caso di sorgente, il tasso di variazione del calore erogato sar\`a sommato al flusso agli estremi.
\[
	\underbrace{\frac{d}{dt}\int_{x_0}^{x_0+h} e(t,x)\rho dx}_\text{Variazione dell'energia rispetto al tempo}
	= \overbrace{q(t,x_0)-q(t,x_0 +h)}^\text{Flusso entrante}
	+\underbrace{\int_{x_0}^{x_0+h} r(t,x) \rho dx}_\text{Flusso della sorgente}
\]
Per il Teorema Fondamentale del Calcolo Integrale
\[
	q(t,x_0)-q(t,x_0 +h) = -\int_{x_0}^{x_0+h} q_x(t,x)dx
\]
dove $q_x$ indica $\frac{dq}{dx}$.

Considerando che l'espressione 
\[
	\frac{d}{dt}\int_{x_0}^{x_0+h} e(t,x)\rho dx
	= -\int_{x_0}^{x_0+h} q_x(t,x)dx
	+\int_{x_0}^{x_0+h} r(t,x) \rho dx
\]
deve essere valida per $x_0$ e $x_0+h$ e che, data la continuit\`a dell'energia \`e possibile portare 
la derivata all'interno del segno di integrale, si ottiene la Legge di Bilancio in forma locale
\[
	\frac{\partial}{\partial t} e(t,x)\rho= -\frac{\partial}{\partial x}q(t,x)+\rho r(t,x)
\]

\`E ora necessario applicare le leggi costitutive, che risultano essere delle leggi sperimentali.\\
La prima, che prende il nome di \textit{Legge di Fourier}, indica che il flusso di calore ($q$) \`e direttamente proporzionale
alla derivata spaziale della temperatura secondo la legge
\[
	q= -ku_x
\]
con $u=u(t,x)$ e $k>0$. Il segno negativo indica che si ha il flusso positivo passando dalla zona pi\`u calda a quella pi\`u fredda.
\[
	[k]= \frac{[cal]}{[tempo]}\frac{[lunghezza]}{[grado]}
\]
La seconda lega invece l'energia alla temperatura
\[
	e= c_lu
\]
dove $c_l$ indica il calore specifico ed \`e $>0$
\[
	[c_l]=\frac{[cal]}{[massa][grado]}
\]
Operando la sostituzione si ottiene
\[
	\rho c_l \frac{\partial}{\partial t} u= k \frac{\partial^2}{\partial x^2}u + \rho r
\]
che riordinata
\[
	u_t= \underbracket{D}_{\mathclap{\text{Risposta termica}}} u_{xx}+f
\]
dove $D=k/c_l\rho$ e $f=r/c_l$.
\[
	[D]=\frac{\cancel{[cal]}[lunghezza]}{[tempo]\cancel{[grado]}}\frac{\cancel{[massa]}\cancel{[grado]}}{\cancel{[cal]}}
	\frac{[lunghezza]}{\cancel{[massa]}}=\frac{[lunghezza]^2}{[tempo]}
\]

L'equazione caratteristica risulta quindi essere, considerata l'equazione differenziale omogenea e sostituendo due variabili algebriche alle due variabili derivate
\[
	u_t= Du_{xx} \;\;\; \Rightarrow \;\;\; T=DX^2
\]
Si noti che \`e l'equazione di una parabola.
%%%%%%%%%%%%%%%%%%%%%%%%%%%%%%%%%%%%%%%%%%%%%%%%%%%%%%%%%%%%%%%%%%%%%%%%%%%%%%%%%%%%%%%%%%%%%%%%%%%%%%%%%%%
\section{Problemi ``Ben Posti''}